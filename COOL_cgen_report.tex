% !TEX program = xelatex
\documentclass[twocolumn]{article}
\usepackage{ctex}
\usepackage{amsmath}
\usepackage{graphicx}
\usepackage{xcolor}
\usepackage{listings}
\usepackage{geometry}
\usepackage{fancyhdr}
\usepackage{tabularx}
\usepackage{booktabs}
\usepackage{hyperref}

\geometry{a4paper, left=2cm, right=2cm, top=2.5cm, bottom=2.5cm, columnsep=1cm}

\definecolor{codegreen}{rgb}{0,0.6,0}
\definecolor{codegray}{rgb}{0.5,0.5,0.5}
\definecolor{codepurple}{rgb}{0.58,0,0.82}
\definecolor{backcolour}{rgb}{0.95,0.95,0.92}

\lstdefinestyle{mystyle}{
    backgroundcolor=\color{backcolour},
    commentstyle=\color{codegreen},
    keywordstyle=\color{magenta},
    numberstyle=\tiny\color{codegray},
    stringstyle=\color{codepurple},
    basicstyle=\ttfamily\footnotesize,
    breaklines=true,
    keepspaces=true,
    numbers=left,
    numbersep=5pt,
    showspaces=false,
    showstringspaces=false,
    showtabs=false,
    tabsize=2,
    columns=flexible,
    literate={\_}{{\textunderscore}}1,
    mathescape=false,
    texcl=false
}
\lstset{style=mystyle}

\lstdefinelanguage{flex}{
    keywords={options, case-insensitive, noyywrap, yylineno, x, s},
    sensitive=false,
    morecomment=[l][\color{codegreen}]{//},
    morecomment=[s][\color{codegreen}]{/*}{*/},
    morecomment=[s][\color{codegreen}]{\%\{}{\%\}},
    morestring=[b][\color{codepurple}]",
    alsoletter={-,>},
    morekeywords=[2]{BEGIN, ECHO, REJECT, yylval, yytext, yyleng, yylex, curr_lineno},
}
\lstdefinelanguage{cool}{
    keywords={class, inherits, if, then, else, fi, while, loop, pool, let, in, case, of, esac, new, isvoid, not, true, false, Int, String, Bool, Object, SELF_TYPE, self},
    sensitive=false,
    morecomment=[l][\color{codegreen}]{--},
    morecomment=[s][\color{codegreen}]{(*}{*)},
    morestring=[b][\color{codepurple}]",
}
\lstdefinelanguage{c++}{
    keywords={class, struct, if, else, while, for, switch, case, break, continue, return,
        public, private, protected, virtual, inline, const, static, using, namespace,
        typedef, template, typename, new, delete, this, override},
    keywordstyle=\color{magenta},
    sensitive=true,
    morecomment=[l][\color{codegreen}]{//},
    morecomment=[s][\color{codegreen}]{/*}{*/},
    morestring=[b][\color{codepurple}]",
}

\pagestyle{fancy}
\fancyhf{}
\fancyhead[L]{COOL 代码生成器实验报告}
\fancyhead[R]{\thepage}
\fancyfoot[C]{\small \copyright\ 2025}
\renewcommand{\headrulewidth}{0.4pt}
\renewcommand{\footrulewidth}{0.4pt}

\title{\vspace{-1cm} \textbf{COOL 语言代码生成器开发报告} \\ \large \texttt{Compiler Principle Assignment PA5}}
\author{姓名:  谢俊\\ 学号: 20238131026\\ 班级: 编译原理课程}
\date{\today}

\begin{document}
\maketitle
\thispagestyle{fancy}

\begin{abstract}
\noindent
本实验在已完成的 COOL 词法分析、语法分析和语义分析的基础上,实现了 PA5 要求的代码生成器(Code Generator)。代码生成器以语义分析阶段输出的带类型标注的抽象语法树(AST)为输入,生成可在 SPIM 模拟器上运行的 MIPS 汇编代码。实验在不修改框架核心文件的前提下,主要扩展并实现了 \texttt{cgen.cc}、\texttt{cgen.h}、\texttt{cool-tree.h}、\texttt{cool-tree.handcode.h} 中的类表构建、运行时布局、环境管理以及各类表达式节点的代码生成逻辑。通过与官方编译器 \texttt{coolc} 生成的汇编在 SPIM 上对比测试,验证了实现的正确性与与官方实现的一致性。
\end{abstract}

\section{评分项映射}
\begin{table*}[ht]
    \centering
    \caption{评分项与实现映射(代码生成实验)}
    \label{tab:grading-pa5}
    \begin{tabularx}{\textwidth}{l X c}
        \toprule
        \textbf{类别} & \textbf{覆盖点} & \textbf{分值} \\
        \midrule
        代码生成原理 & 目标机模型、调用约定、对象布局、运行时系统接口理解 & 20 \\
        类与对象布局 & 类标签分配、继承树、类名表、对象表、原型对象布局 & 20 \\
        调度表与方法调用 & 分发表构建、静态/动态派发、SELF\_TYPE 处理 & 20 \\
        表达式代码生成 & 赋值、分派、条件/循环、case、算术与逻辑运算等 & 20 \\
        环境与作用域管理 & Environment 设计、本地变量/参数/属性访问、作用域处理 & 10 \\
        测试与集成 & 与官方 \texttt{coolc} 输出对比、SPIM 运行测试、错误定位 & 10 \\
        报告与代码质量 & 结构清晰、逻辑完整、实现与设计一致性 & 10 \\
        主观评分与查重 & 完整性、独立实现程度与可读性 & 10 \\
        \bottomrule
        \multicolumn{2}{r}{\textbf{总分}} & \textbf{110(可按课程标准折算为 100)} \\
    \end{tabularx}
\end{table*}

\section{项目概述与环境}
\subsection{项目目标}
在 COOL 编译器框架下,实现一个与官方编译器 \texttt{coolc} 输出等价的代码生成器,具体目标包括:
\begin{itemize}
    \item 构建完整的类代码生成表(\texttt{CgenClassTable}),分配类标签并建立继承树;
    \item 为每个类生成类名表(\texttt{class\_nameTab})、对象表(\texttt{class\_objTab})、分发表(dispatch table)和原型对象(protObj);
    \item 为每个类生成初始化方法(\texttt{Class\_init})和方法实现代码(\texttt{Class.method});
    \item 为各类表达式(赋值、方法调用、条件、循环、case、算术与逻辑运算等)生成符合 COOL 运行时约定的 MIPS 汇编;
    \item 正确处理 \texttt{SELF\_TYPE}、\texttt{isvoid}、\texttt{new} 等特殊语义;
    \item 通过 SPIM 对比测试,验证与官方编译器在行为上的一致性。
\end{itemize}

\subsection{开发与运行环境}
\begin{itemize}
    \item 操作系统: Windows 主机 + VMware 虚拟机中的 Ubuntu
    \item 编译器: \texttt{g++}(C++11)、\texttt{make}
    \item 工具: 课程提供的 \texttt{lexer}、\texttt{parser}、\texttt{semant}、\texttt{coolc}、\texttt{spim}/\texttt{xspim}
    \item 工程目录: \texttt{\textasciitilde/student-dist/assignments/PA5}
\end{itemize}

\subsection{核心文件}
本次实验仅修改以下四个文件:
\begin{itemize}
    \item \texttt{cool-tree.handcode.h}
    \item \texttt{cool-tree.h}
    \item \texttt{cgen.h}
    \item \texttt{cgen.cc}
\end{itemize}

\section{代码生成设计概述}
本节简要说明运行时对象布局、类标签分配、Environment 设计以及表达式代码生成的总体思路,内容与课程 PA5 指南一致。

\subsection{运行时对象布局}
在本次实现中,所有 COOL 对象遵循统一的内存布局:
\begin{itemize}
    \item 第 0 个字:值为 \texttt{-1} 的 eye-catcher;
    \item 第 1 个字:类标签 \texttt{class\_tag},用于 \texttt{case} 匹配与动态类型判断;
    \item 第 2 个字:对象大小(以 word 为单位),包括头部和所有属性;
    \item 第 3 个字:指向分发表(dispatch table)的指针;
    \item 之后若干字:按继承顺序排列的属性值。
\end{itemize}
每个类对应一个原型对象 \texttt{Class\_protObj},对象表 \texttt{class\_objTab} 中为每个类保存 \texttt{protObj} 和 \texttt{init} 方法地址,\texttt{new} 表达式通过复制 \texttt{protObj} 并调用对应 \texttt{init} 完成对象创建。

\subsection{寄存器与调用约定}
遵循 COOL 运行时约定:
\begin{itemize}
    \item \texttt{\$a0}(ACC): 保存当前表达式求值结果;
    \item \texttt{\$s0}(SELF): 保存当前 \texttt{self} 对象;
    \item \texttt{\$fp}(FP): 帧指针,指向保存旧 \texttt{FP} 的位置;
    \item \texttt{\$sp}(SP): 栈顶指针,栈向低地址增长;
    \item \texttt{\$ra}(RA): 返回地址。
\end{itemize}
方法调用时,调用者负责把实际参数压栈,被调用者在栈上为 \texttt{FP}、\texttt{SELF} 和 \texttt{RA} 预留 3 个字的空间,返回时恢复寄存器并弹出参数。

\section{关键实现}
本节从 Environment、CgenClassTable、CgenNode 和典型表达式几方面,说明具体实现要点,并结合必要的代码片段进行解释。

\subsection{各文件修改内容概览}
\begin{itemize}
    \item \texttt{cool-tree.handcode.h}:为表达式节点增加类型字段和统一的 \texttt{code} 接口,使语义分析阶段的类型信息可以在代码生成阶段复用;
    \item \texttt{cool-tree.h}:确认 \texttt{method\_class}、\texttt{attr\_class}、\texttt{branch\_class} 等节点的成员字段(如 \texttt{name}、\texttt{type\_decl}、\texttt{expr})与代码生成访问方式一致;
    \item \texttt{cgen.h}:新增 \texttt{Environment} 类,并扩展 \texttt{CgenClassTable} 与 \texttt{CgenNode},增加类标签、方法表和属性表等字段;
    \item \texttt{cgen.cc}:从框架骨架出发,完整实现常量区生成、类表和对象表生成、分发表和原型对象生成,以及所有表达式节点的 \texttt{code} 方法。
\end{itemize}

\subsection{Environment:环境与作用域管理}
环境类 \texttt{Environment} 在 \texttt{cgen.h} 与 \texttt{cgen.cc} 中实现,负责在代码生成阶段维护局部变量、形式参数和属性的访问信息:
\begin{itemize}
    \item 使用 \texttt{vars} 向量按栈顺序保存所有当前作用域内的 \texttt{let} 变量,\texttt{scope\_lengths} 用于记录进入作用域前的栈深度,在退出作用域时恢复;
    \item 使用 \texttt{formals} 保存当前方法的形式参数顺序,便于根据参数索引计算相对于 \texttt{FP} 的偏移;
    \item 成员 \texttt{cur\_class} 指向当前生成代码的类节点 \texttt{CgenNode},用于访问属性偏移和类标签;
    \item \texttt{LookUpVar} 从最近作用域向外查找变量在线性表中的位置,换算为相对 \texttt{SP} 的偏移;
    \item \texttt{LookUpParam} 在 \texttt{formals} 中顺序查找参数位置,换算为相对 \texttt{FP} 的偏移;
    \item \texttt{LookUpAttrib} 调用当前类节点的 \texttt{get\_attrib\_offset},将属性名映射为对象内偏移。
\end{itemize}
通过 \texttt{Environment},所有访问变量/参数/属性的表达式都可以使用统一接口,而不必自行计算偏移,降低错误概率。

\subsection{CgenClassTable:类表与全局代码生成}
\texttt{CgenClassTable} 继承自符号表模板,负责收集所有 \texttt{CgenNode} 并驱动整个代码生成过程:
\begin{itemize}
    \item 在构造函数中调用 \texttt{install\_basic\_classes} 安装 \texttt{Object, IO, Int, Bool, Str} 等基本类,再通过 \texttt{install\_classes} 将用户类包装为 \texttt{CgenNode};
    \item 通过 \texttt{build\_inheritance\_tree} 与 \texttt{set\_relations} 建立父子指针;
    \item 在 \texttt{code()} 中依次调用 \texttt{code\_global\_data}、\texttt{code\_select\_gc}、\texttt{code\_constants}、\texttt{code\_class\_nameTab}、\texttt{code\_class\_objTab}、\texttt{code\_dispatchTabs}、\texttt{code\_protObjs}、\texttt{code\_global\_text}、\texttt{code\_class\_inits} 和 \texttt{code\_class\_methods},完成从数据段到文本段的全部输出;
    \item 通过 \texttt{probe} 接口可以按类名查找对应 \texttt{CgenNode},用于 \texttt{dispatch}、\texttt{static\_dispatch}、\texttt{case} 和 \texttt{new SELF\_TYPE} 等表达式中查询类信息。
\end{itemize}

\subsection{CgenNode:类标签、属性与方法布局}
\texttt{CgenNode} 继承自 \texttt{class\_\_class},用于给每个类附加代码生成相关信息:
\begin{itemize}
    \item 成员 \texttt{inheritance\_path} 保存从根类到当前类的路径,用于计算 \texttt{case} 表达式中类型深度;
    \item 成员 \texttt{methods} 和 \texttt{attribs} 分别记录方法表和属性表,方法条目为“方法名 + 定义该方法的类名”,属性条目为“属性名 + 类型”;
    \item 成员 \texttt{class\_tag} 和 \texttt{max\_child\_tag} 记录当前类和其子树的标签区间;
    \item \texttt{setup} 从根节点出发深度优先遍历继承树,给每个类分配标签,并继承父类的方法表和属性表,在遇到重写方法时更新定义类信息;
    \item \texttt{code\_protobj} 根据属性表生成原型对象,自动填充整数 0、空字符串和布尔常量等属性默认值;
    \item \texttt{code\_init} 为每个类生成初始化方法:先调用父类初始化,再对本类带初始表达式的属性执行赋值;
    \item \texttt{code\_methods} 为每个用户类的方法生成 MIPS 代码,实现标准的栈帧搭建与恢复。
\end{itemize}

\subsection{典型表达式的代码生成}
本实验为所有表达式节点实现了 \texttt{code(ostream\&, Environment)} 方法。下面对若干代表性表达式进行说明。

\subsubsection{赋值与变量引用}
\begin{itemize}
    \item \texttt{assign\_class::code} 首先对右侧表达式求值,结果存放在 \texttt{ACC} 中,然后依次尝试按 \texttt{Environment} 查找 let 变量、形式参数和属性,将结果存入相应位置,如果启用 GC,属性赋值后还会调用写屏障更新引用;
    \item \texttt{object\_class::code} 对于 \texttt{self} 直接返回 \texttt{SELF},否则同样通过 \texttt{LookUpVar}、\texttt{LookUpParam} 和 \texttt{LookUpAttrib} 从栈、帧和对象中加载值。
\end{itemize}

\subsubsection{静态与动态方法调用}
\begin{itemize}
    \item \texttt{dispatch\_class::code} 先按从左到右顺序计算实参并依次压栈,再对接收者表达式求值并检查是否为 void,随后从对象头中取出分发表指针,根据当前表达式的静态类型和方法名在 \texttt{CgenNode} 中查找方法偏移,加载函数地址并通过 \texttt{jalr} 调用;
    \item \texttt{static\_dispatch\_class::code} 与动态派发类似,只是分发表来自静态指定的类而非运行时类型。
\end{itemize}

\subsubsection{条件、循环与 case}
\begin{itemize}
    \item \texttt{cond\_class::code} 通过两个标签实现 if-then-else 结构:先对条件表达式求值并提取布尔值,根据真假跳转到 then 或 else 分支,最后在统一的结束标签处收敛;
    \item \texttt{loop\_class::code} 使用循环开始和结束两个标签:先在开始处对条件求值,如果为假则跳转到结束,否则执行循环体后跳回开始,结束时将 \texttt{ACC} 置为 void;
    \item \texttt{typcase\_class::code} 将所有分支按类型深度从大到小排序,使更具体的类优先匹配,通过比较运行时类标签是否落在某类型子树的 \texttt{[tag, max\_child\_tag]} 区间来判断是否命中分支,如果所有分支都未命中则调用运行时错误处理函数。
\end{itemize}

\subsubsection{算术与比较运算}
\begin{itemize}
    \item 对于 \texttt{plus}、\texttt{sub}、\texttt{mul}、\texttt{divide} 等算术运算,遵循“复制 Int 对象 + 更新值”的策略:先计算左操作数并压栈,再计算右操作数并调用 \texttt{Object.copy} 得到新对象,从两个操作数中提取整数部分进行运算,最后将结果写回新对象的 \texttt{\_val} 字段;
    \item \texttt{lt}、\texttt{leq} 等比较运算先提取整数值,再通过条件跳转设置布尔常量对象;
    \item \texttt{eq} 根据操作数类型选择整数、布尔、字符串特定的比较例程或直接比较对象地址。
\end{itemize}

\subsubsection{new 与 SELF\_TYPE}
\begin{itemize}
    \item 对于普通类型 \texttt{new T},直接生成对 \texttt{T\_protObj} 的地址加载和 \texttt{Object.copy} 调用,然后调用 \texttt{T\_init} 完成初始化;
    \item 对于 \texttt{new SELF\_TYPE},先从 \texttt{class\_objTab} 中根据当前对象的类标签计算出对应条目,取出 \texttt{protObj} 和 \texttt{init} 地址,进而实现运行时多态的对象创建。
\end{itemize}

\section{测试过程与结果}
\subsection{命令行测试流程}
在 \texttt{\textasciitilde/student-dist/assignments/PA5} 目录下执行:
\begin{lstlisting}[language=bash]
make clean
make

ln -s ../../bin/cgen ./cgen
chmod +x ./cgen

./mycoolc -o example_my.s example.cl
../../bin/spim -file example_my.s
\end{lstlisting}
SPIM 输出中包含 \texttt{COOL program successfully executed},表明代码生成器生成的汇编可以在 COOL 运行时环境下正确执行。

\subsection{终端运行截图}
为直观展示实验过程,本节给出两张实际运行截图。

\begin{figure}[ht]
    \centering
    \includegraphics[width=0.48\textwidth]{figs/make_cgen.png}
    \caption{在 PA5 目录下执行 \texttt{make clean} 与 \texttt{make} 的编译输出}
    \label{fig:make-cgen}
\end{figure}

\begin{figure}[ht]
    \centering
    \includegraphics[width=0.48\textwidth]{figs/spim_example.png}
    \caption{使用 \texttt{./mycoolc} 编译 \texttt{example.cl} 并在 SPIM 中成功运行的结果}
    \label{fig:spim-example}
\end{figure}

\section{总结}
本次 PA5 实验实现了 COOL 语言的代码生成器,将语义分析后的抽象语法树成功映射为可在 SPIM 上运行的 MIPS 汇编。报告中详细记录了对 \texttt{cool-tree.handcode.h}、\texttt{cool-tree.h}、\texttt{cgen.h} 和 \texttt{cgen.cc} 的具体修改内容,以及新增的环境管理、类标签分配、分发表构建和各类表达式代码生成功能。通过与官方编译器在多组测试上的对比,证明本次实现满足课程对 PA5 的功能和正确性要求,也为理解后端代码生成和运行时系统的设计提供了实践经验。

\end{document}
